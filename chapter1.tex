\documentclass{article}
\usepackage{amsthm}
\usepackage{amsfonts}
\usepackage{enumitem}

\title{Introduction}
\author{xuascaler}
\date{\today}

\newtheorem*{property}{Property}
\newtheorem*{definition}{Definition}
\newtheorem*{remark}{Remark}
\newtheorem*{example}{Example}
\newtheorem*{corollary}{Corollary}
\newtheorem*{theorem}{Theorem}

\begin{document}
\maketitle

\section*{What is Functional analysis? A brief introduction}
Functional analysis is known as modern analysis or soft analysis.
\begin{itemize}
    \item In classical analysis (like mathmatical analysis, real analysis, complex analysis)
    We mainly concern the quantitative properties of individual functions.
    (e.g. the value, the integral, the derivative, estimate)
    \item In functional analysis we study qualitative properties in space of functions.
    We will treat functions as "points" in some sort of abstract space, e.g.
    \[
        L_2(\mathbb{R}) = \textnormal{the space of all square-integral functions on } \mathbb{R}
    \]
\end{itemize}
Functional analysis is infinite dimensional linear algebra.
\begin{itemize}
    \item In (finite dimensional) linear algebra we study properties of $\mathbb{R}_n$ and 
    linear maps from $\mathbb{R}^n$ to $\mathbb{R}^m$(or $\mathbb{C}^n$ to $\mathbb{C}^m$).
    For example, one of the major theorems in linera algebra is
    \begin{theorem}
        Let $\mathcal{F}:\mathbb{R}^n \rightarrow \mathbb{R}^n$ be a linear map whose matrix is symmetric.
        then we can find orthogonal set of vectors $\overrightarrow{v_{1}}, ......, \overrightarrow{v_{n}}$
        and scalars $\lambda_1,......,\lambda_n$, so that $F\overrightarrow{v_i}=\lambda_{i}\overrightarrow{v_{i}}$
        In particular, any vector $\overrightarrow{v} \in \mathbb{R}_{n}$ can be written as
        $\overrightarrow{v} = \alpha_1\overrightarrow{v_1} + ...... + \alpha_n\overrightarrow{v_n}$
        and we have 
        $F\overrightarrow{v} = \alpha_1\lambda_1\overrightarrow{v_1} +......+\alpha_n\lambda_n\overrightarrow{v_n}$.
    \end{theorem}
    \item In (linear) functional analysis we study properties of infinite dimensional vector spaces,
    and "linear maps" between them. For example,
    \begin{enumerate}
        \item $L^2(\mathbb{R}) \textnormal{ is a vector space: } f, g \in L_2(\mathbb{R}) \Rightarrow \lambda f + \mu g \in L_2(\mathbb{R})$
        \item The map $\int : L^1(\mathbb{R}) \rightarrow \mathbb{R}, f \rightarrow \int_{\mathbb{R}}{|f(x)|}dx$ is a linear map.
    \end{enumerate}
    We will see how to extend the above theorem to infinite dimensional case.
\end{itemize}
Functional analysis was developed in early 20 centuries, 
origined by problems in caculus of variation, 
integral equations and quantum physics. 
It was deeply influenced by the idea of axiomatization at that time.
It has vast applications in PDE, probability, computational math, applied math and in mathmatical physics.
Note: We mainly work on linear functional analysis.

\section*{1 Infinite dimensional vector spaces}
\begin{remark}
    A vector space(over $\mathbb{R}$ or $\mathbb{C}$) V is a set, together with two operations:
    \begin{itemize}
        \item $+: V \times V \rightarrow V$
        \item $\cdot:\mathbb{R}(\textnormal{or } \mathbb{C}) \times V \rightarrow V$
    \end{itemize}
    such that all the properties that we are familiar with hold, e.g.
    \begin{itemize}
        \item $x + y = y + x$
        \item $(x + y) + z = x + (y + z)$
        \item $x + 0 = x$
        \item $x + (-x) = 0$
        \item $\alpha(x + y) = \alpha x + \alpha y$
        \item $(\alpha + \beta)x = \alpha x + \beta x$
        \item $1 \cdot x = x$
        \item $(\alpha \beta)x = \alpha(\beta x)$
    \end{itemize}
    A set of vectors ${u_1, ......, u_n}$ is called the basis of $V$ 
    if every $x \in V$ can be written uniquely as 
    $x = \alpha_1\mu_1 + ..... + \alpha_n\mu_n$
    In this case we say $dimV = n$
    If $V$ has no finite basis, tha we say $V$ is infinite dimensional.
    \begin{example}
        $\mathcal{P} = \textnormal{the set of all polynomials in variable } x$.
        Easy to see:
        \begin{itemize}
            \item $\mathcal{P}$ is a vector space.
            \item $\mathcal{P}$ has no finite basis.
            \begin{proof}
                If polynomials $p_1, ......, p_n$ is any set of polynomials. 
                Let $N=max\{deg(p_1), ......, deg(p_n)\}$.
                Then any polynomials of degree $N + 1$ is NOT a linear combination of $p_i$'s.
                So ${p_1, ......, p_n}$ is NOT a basis of $\mathbb{P}$.
                So $V$ is an infinite dimensional vector space.
            \end{proof}
        \end{itemize}
    \end{example}
    A Hamel basis for a vector space $V$ is a set of linearly independent vectors in $V$ so that every $x \in V$
    can be wriiten uniquely as a finite linear combination of elements in this set.
    \begin{example}
        $\{1, x, x_2, ......, x_n\}$ is Hamel basis of $\mathcal{P}$ in previous example.
    \end{example}
    Every vector space admits a Hamel basis (by the axiom of choice).(see proof in appendix)
    But for most interesting infinite demensional vector spaces, its Hamel basis has uncountable cardinality.
    There are other conceptions of basis for infinitely dimensional vector spaces.
    We will not study them in this course, except for Hilbert spaces.
    \begin{example}
        \begin{itemize}  \hfill
            \item $C^{\infty}(\mathbb{R}), L^2(\mathbb{R}), C^{\infty}(\bigl[0, 1\bigr]), L^2(\bigl[0, 1\bigr])$ etc.
            They all contain $\mathcal{P}$ in the previous example, with the exception $L^2(\mathbb{R})$.
            But for any $p(x) \in \mathcal{P}, p(x)e^{-x^2} \in L^2(\mathbb{R})$.
            \item $l_2 = \{x=(a_1, a_2, ......, a_n)\mid a_i \in \mathbb{R}, \sum_{i=1}^{\infty}a_i^2 < \infty\}$
        \end{itemize}
    \end{example}
    As in the course linera algebra, one can define the conception of vector subspace and prove theorems like
    "the intersectoin of vector subspace is a vector space".
\end{remark}
\section*{2 Metric structures on vector spaces}
We would introduce distance between elements in an abstract vector space.
\begin{definition}
    A metric on a vector space $X$ is a real-valued function $d: X \times X \rightarrow \mathbb{R}$ so that
    \begin{enumerate}
        \item (positivity) $d(x, y)\ge0$ with equality iff $x=y$.
        \item (symmetry) $d(x, y)=d(y, x)$
        \item (triangle inequality) $d(x, y) \le d(x, z) + d(z, y)$
    \end{enumerate}
    \begin{remark}
        One can define metric on any set.
    \end{remark}
    \begin{example}
        $X=C(\bigl[0, 1\bigr])$, the space of all continuous functions on $\bigl[0, 1\bigr]$
        \begin{itemize}
            \item $d_0(f, g) = max_{0 \le x \le 1}|f(x) - g(x)|$ is a metric.
            \item $d_1(f, g) = \bigl(\int_{0}^{1}{|f(x) - g(x)|^2}\bigr)^\frac{1}{2}$ is also a metric.
        \end{itemize}
    \end{example}
\end{definition}
In mathmatics, usually compatibale structures are more interesting.
$\mathbb(Q)$: What are the metrics on $X$ that are compatibale with vector space operations?
\begin{definition}
    A metric on a vector space $X$ is called translation-invariant if 
    $d(x + z, y + z) = d(x, y), \forall x, y, z \in X$.
\end{definition}
Obviously the distances in the above examples are translation-invariant, positive homogeneity.
\begin{definition}
    A metric on a vector space $X$ is called positively homogeneous if
    $d(\alpha x, \alpha y) = |\alpha|d(x, y)$.
\end{definition}
Now suppose $X$ is a vector space and $d(\cdot, \cdot)$ is a metric compatibale 
with $+$ and $\cdot$, i.e.
\begin{enumerate}[label=\alph*)]
    \item $d(x + z, y + z) = d(x, y)$
    \item $d(\alpha x, \alpha y) = |\alpha|d(x, y)$
\end{enumerate}
We define a function $\|\cdot\|:X \rightarrow \mathbb{R}$ by $\|x\|=d(x, 0)$.
\begin{property}
    The function $\|\cdot\|$ satisfies
    \begin{enumerate}
        \item $\|x + y\| \le \|x\| + \|y\|$
        \item $\|\alpha x\| = |\alpha|\|x\|$
        \item $\|x\| > 0 \textnormal{ if } x \ne 0$
    \end{enumerate}
    \begin{proof} \hfill
        \begin{itemize}
            \item $\|x + y\|=d(x+y, 0) \le d(x+y, y) + d(y, 0) = d(x, 0) + d(y, 0) = \|x\| + \|y\|$
            \item $\|\alpha x\| = d(\alpha x, \alpha 0) = |\alpha|d(x, 0) = |\alpha|\|x\|$
            \item if $x \ne 0$, then $\|x\|=d(x, 0) > 0$
        \end{itemize}
    \end{proof}
\end{property}
\begin{definition}
    Let $X$ be a vector space. A function $\|\cdot\|: X \rightarrow \mathbb{R}$ is called a norm
    if it satisfies the property (1), (2), (3), we call the pair $\bigl(X, \|\cdot\|\bigr)$ a normed vector space.
\end{definition}
\begin{property}
If $(X, \|\cdot\|)$ is a normed vector space, then the function $d(x, y):=\|x-y\|$
defines a metric a $X$ which is compatibale with vector space structures.
i.e. d satisfies (1), (2), (3), (a), (b).
\begin{proof}
    Exercise.
\end{proof}
\end{property}
\textbf{normed vector space = vector space with a compatible metric.}
\end{document}