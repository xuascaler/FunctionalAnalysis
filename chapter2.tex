\documentclass{article}
\usepackage{amsmath, amsthm}
\usepackage{amsfonts}
\usepackage{enumitem}

\title{Complete metric spaces}
\author{xuascaler}
\date{\today}

\newtheorem*{property}{Property}
\newtheorem*{definition}{Definition}
\newtheorem*{remark}{Remark}
\newtheorem*{example}{Example}
\newtheorem*{corollary}{Corollary}
\newtheorem*{theorem}{Theorem}

\begin{document}
\maketitle
\section*{1 Convergence}
Let $(X, d)$ be a metric space(not necessary to be a vector space).
\begin{definition}
    We say that a sequence of vectors, $\{x_1, x_2, ....., x_n\}$, in $X$ converges to $x_0 \in X$
    (under the metric d)
    if $\forall \epsilon > 0, \exists N > 0$ s.t. $\forall i > N, d(x_i, x_o)<\epsilon$
    We will introduce Convergence in more general spaces later.
    \begin{example}
        $X=C(\bigl[0, 1\bigr])$
        \begin{enumerate}
            \item $d_1(f, g) = \int_{0}^{1}{|f(x) - g(x)|dx}$ \\
            Let $f_n(x) = x_n, f_0(x)=0$. 
            Then $d_1(f_n, f_0) = \int_{0}^{1}{x_n}dx = \frac{1}{n+1} \rightarrow 0 \textnormal{ as } n \rightarrow \infty$. \\
            so $f_n \rightarrow f_0$ in $(X, d_1)$
            \item $d_0(f, g) = \max_{0 \le x \le 1}|f(x) - g(x)|$
            Still take $f_n(x) = x_n, f_0(x) = 0$, Then
            $d_0(f_n, f_0) = \sup_{0 \le x \le 1}{x_n} = 1$,
            so $f_n \not\rightarrow f_0$ in $(X, d_0)$.
            \begin{remark}
                Obvious one always have $d_1(f, g) \le d_0(f, g) in X$
                We say the metric $d_0$ is stronger than $d_1$. If a sequence converges to an 
                element in $d_0$, then it converges to that element in $d_1$.
            \end{remark}
            \item Still take $d_1(f,g) = \int_{0}^{1}{|f(x)-g(x)|dx}$. 
            \[
                f_n(x) = 
                \begin{cases}
                    0, & x < \frac{1}{2} - \frac{1}{n}, \\
                    \frac{1}{2} + \frac{n}{2}(x - \frac{1}{2}), & \frac{1}{2} - \frac{1}{n} \le x \le \frac{1}{2} + \frac{1}{n}, \\
                    1, & x > \frac{1}{2} + \frac{1}{n}
                \end{cases}
            \]
            and let \[
                \begin{cases}
                    0, & x \le \frac{1}{2}, \\
                    1, & x \ge \frac{1}{2}
                \end{cases}
            \] 
            Then $d_1(f_n, f_0) = \int_{0}^{1}{|f_n(x) - f_0(x)|dx} = \frac{1}{2n} \rightarrow 0$.
        \end{enumerate}
    \end{example}
\end{definition}
\begin{property}
Suppose a sequence $\{x_n\}$ converges in $(X, d)$, then
\begin{enumerate}
    \item $\{x_n\}$ is bounded.
    \item the limit is unique.
\end{enumerate}
\begin{proof} \hfill
    \begin{enumerate}
        \item Suppose $x_n \rightarrow x_0$, then for $\epsilon = 1, \exists N$, 
        s.t. $\forall i > N$, $d(x_n, x_0) < 1$.
        Let $C = max(d(x_n, x1), d(x_n, x_1),......,d(x_n, x_0)) + 1$,
        Then $d(x_n, x_0) < C \forall 1 \le i < \infty$
        So $\{x_i\}$ is bounded.
        \item Suppose $x_i \rightarrow x_0$ and $x_i \rightarrow x_0'$.
        $\forall \epsilon > 0, \exists N, N'$ s.t.
        $\forall i > N, d(x_i, x_0) < \epsilon, \forall i' > N, d(x_i', x_0') < \epsilon \Rightarrow$
        $d(x_0, x_0') \le d(x_0, x_i) + d(x_i, x_0') \le 2\epsilon \Leftrightarrow d(x_0, x_0') = 0$
        $\Rightarrow x_0 = x_0'$
    \end{enumerate}
\end{proof}
\end{property}
\section*{2 Completeness}
As we have seen in part 3 of previous example, we have a sequence in $X$ which converges under $d_1$ to
, an element outside $X$. So as $\mathcal{Q}, (X, d)$ is NOT complete. To do better analysis, we would 
like to work on complete spaces. As in mathmatical analysis, we define
\begin{definition}
    A sequence $\{x_i\}$ in $(X, d)$ is a Cauchy sequence if $\forall \epsilon > 0, \exists N > 0$
    s.t. $\forall i, j > N$, $d(x_i, x_j) < \epsilon$
\end{definition}
\begin{definition}
    A metric space $(X, d)$ is complete if any Cauchy sequence in $(X,d)$ converges to an element in $X$.
\end{definition}
\begin{definition}
    A complete normed vector space is called a Banach space.
    \begin{itemize}
        \item Here, the metric is the induced metric from the norm: $d(x, y) = \|x - y\|$
        \item Banach space will be one of the main object in this course.
    \end{itemize}
    \begin{example}
        $d(x, y) = (\sum_{i=1}^{n}{(x_i-y_i)})^\frac{1}{2}$ is a complete metric on $\mathbb{R}^n$.\\
        $\|x\|=(\sum_{i=1}^{n}{x_i^2})^\frac{1}{2}$, $(\mathbb{R}^n, \|\cdot\|)$ is a Banach space.
    \end{example}
    \begin{example}
        $X=C(\bigl[0, 1\bigr])$.
        \begin{enumerate}
            \item $d_1(f, g) = \int_{0}^{1}{|f(x) - g(x)|dx}$.
            We have seen $f_n(x) = x^n \rightarrow f_0(x) = 0$.
            In fact, $\{f_n\}$ is a Cauchy sequence, since
            $d(f_n, f_m) = \int_{0}^{1}|x^n - x^m|dx < \int_{0}^{1}{x^ndx} + \int_{0}^{1}{x^mdx}=\frac{1}{n+1} + \frac{1}{m+1}$
            In general, we have
            \begin{property}
                Any converged sequence in a metric space is a Cauchy sequence.
                \begin{proof}
                    Suppose $x_i \rightarrow x_0$. i.e. $\forall \epsilon > 0, \exists N$ s.t.
                    $\forall i > N, d(x_i, x_0) < \epsilon$.
                    So for $\forall i, j > N$, we have
                    $d(x_i, x_j) \le d(x_i, x_0) + d(x_0, x_j) < 2\epsilon$.
                    So $\{x_i\}$ is Cauchy.
                \end{proof}
            \end{property}
            \item $d_0(f, g) = \max_{0 \le x \le 1}{|f(x) - g(x)|dx}$.\\
            We have seen $f_n \not\rightarrow f_0$ in $(X, d_0)$.
            In fact, $\{f_n\}$ is not a Cauchy sequence since we fix $n$ and let $m \rightarrow \infty$,
            $d(f_n, f_m) = \max_{0 \le x \le 1}{|x_n - x_m|dx} \rightarrow 1$.
            In fact, we have 
            \begin{property}
                $(C(\bigl[0, 1\bigr]), d_0)$ is complete. As a consequence, \\
                $\|f\|_0=\sup_{0 \le x \le 1}{|f(x)|}, (C(\bigl[0, 1\bigr]), \|f_0\|)$ is a Banach space.
                \begin{proof}
                    Let $\{f_n\}$ be a Cauchy sequence, $\forall \epsilon > 0, \exists N > 0$, s.t.
                    $d_0(f_n, f_m) = \sup_{0 \le x \le 1}|f_n(x) - f_m(x)| < \epsilon$.
                    Then for any fixed $x \in [0, 1]$, the sequence (of scalars) $\{f_n(x)\}$ is a Cauchy sequence in R.
                    It follows that there exists $f_0(x)$ s.t. $f_n(x) \rightarrow f_0(x)$(use Completeness of $\mathbb{R}$).
                    Since $|f_n(x) - f_m(x)| < \epsilon$. letting $m \rightarrow \infty$ we get
                    $|f_n(x) - f_0(x)| \le \epsilon, \forall n > N, \forall x \in X$.
                    So the sequence of functions $\{f_n(x)\}$ converges uniformly to $f_0(x)$, 
                    because $\forall x \in X$.
                    By results in mathmatical analysis, $f_0$ is continuous and 
                    $f_n \rightarrow f_0(\sup_{0 \le x \le 1}{|f_n(x) - f_0(x)|} < \epsilon, \forall n > N)$ 
                    in $(C([0, 1]), d_0)$
                    (Finally, since each \(f_n\) is continuous and the uniform limit of continuous functions is continuous, 
                    \(f\in C([0,1])\).
                    We have thus found a limit \(f\in C([0,1])\) of the Cauchy sequence \(\{f_n\}\) in the metric \(d_0\).  
                    This shows \((C([0,1]),d_0)\) is complete. See the proof in appendix)
                \end{proof}
            \end{property}
            \item \[
                d_1(f, g) = \int_{0}^{1}{|f(x) - g(x)|dx}
            \]
            \[
                f_n(x) = 
                \begin{cases}
                    0, & x < \frac{1}{2} - \frac{1}{n}, \\
                    \frac{1}{2} + \frac{n}{2}(x - \frac{1}{2}), & \frac{1}{2} - \frac{1}{n} \le x \le \frac{1}{2} + \frac{1}{n} \\
                    1, & x > \frac{1}{2} + \frac{1}{n}
                \end{cases}
            \]
            \[
                f_0(x) =
                    \begin{cases}
                        0, & x \le \frac{1}{2} \\
                        1, & x \ge \frac{1}{2}
                    \end{cases}
            \] 
            $f_n$ is actually a Cauchy sequence in $(X, d_1)$, since \\
            $d_1(f_n, f_m) \le d_1(f_n, f_0) + d_1(f_0, f_m) \le \frac{1}{2n} + \frac{1}{2m}$.
            Conclusion: $(C([0, 1]), d_1)$ is NOT complete.
            \begin{remark}
                $C([0, 1]) \subset L^1([0, 1])$ and we will see $(L^1, d_1)$ is complete.
            \end{remark}
            In general, for any incomplete metric space $(X, d)$, it is possible to construct a complete metric space
            $(\overline{X}, \overline{d})$ so that $X$ is dense(we will define this next time) in $\overline{X}$ and
            $\overline{d}|_X = d$.
            The procedure is the same as $\mathbb{Q} \rightarrow \mathbb{R}$.
            See HW next time.
            \item $X = l_1 = {\mathbf{x}=(a_1, a_2, ......)\mid\sum_{i=1}^{\infty}{|a_i| < \infty}}$.
            $\|\mathbf{x}\| = \sum_{i=1}^{\infty}{|a_i|}$. Then $(X, \|\cdot\|)$ is a Banach space.
            \begin{proof} \hfill
                \begin{itemize}
                    \item $X$ is a vector space because for 
                    $\mathbf{x}=(a_1, a_2, a_3, ...), \mathbf{y}=(b_1, b_2, b_3, ...)$,
                    \[
                    x + y \in X: \sum_{i=1}^{\infty}{a_i + b_i} \le \sum_{i=1}^{\infty}{|a_i|} + \sum_{i=1}^{\infty}{|b_i|} < \infty
                    \]
                    \[
                    \alpha x \in X: \sum_{i=1}^{\infty}{\alpha a_i} = |\alpha|\sum_{i=1}^{\infty}{|a_i| < \infty}
                    \]
                    The axioms hold in an obvious way.
                    \item $\|\cdot\|$ is norm since
                    \begin{itemize}
                        \item $\|x + y\| \le \|x\| + \|y\|$
                        \item $\|\alpha x\| = \alpha \|x\|$
                        \item if $x \ne 0$, then $\exists a_i \ne 0$. So $\|x\| = \sum_{i=1}^{\infty}{|a_i|} > 0$
                    \end{itemize}
                    \item Completeness:
                    Let $x^j = (a_i^j)$ be a Cauchy sequence in $l^1$, i.e.
                    $\forall \epsilon > 0, \exists N$ s.t. $\forall i, k \ge N, \|x^j - x^k\| = \sum_{l}|a_l^j - a_l^k| < \epsilon$.
                    So $\forall l$ are fixed, $\forall j, k > N, |a_l^j - a_l^k| < \epsilon$
                    \[
                       \Rightarrow \forall l \textnormal{ are fixed }, \{a_l^j\} \textnormal{ is a Cauchy sequence in }\mathbb{R} 
                    \]
                    \[
                        \Rightarrow \exists a_l^0 \in \mathbb{R} \textnormal{ s.t. } a_l^j \xrightarrow{j \rightarrow \infty} a_l^0.
                    \]
                    We want to show that $x_0=(a_l^0) \in l^1$, and $x^j \rightarrow x^0$ in $(X, d)$ with $d(x, y) = \|x -y\|, $
                    To prove this, we choose $M$ large so that $\sum_{i=M}^{\infty}{|a_i^N|} < \epsilon$
                    Then for $j > N$, we have
                    \[
                        \sum_{i=M}^{\infty}{|a_i^j|} \le \sum_{i=M}^{\infty}{|a_i^j - a_i^N| + \sum_{i=M}^{\infty}{|a_i^N|}} < 2 \epsilon
                    \]
                    letting $j \rightarrow \infty$, we get $\sum_{i=M}^{\infty}{a_i^0} < 2 \epsilon$. so $x^0 \in l^1$.
                    Moreover, choose $j > N$ large enough, we can get
                    \[
                        \sum_{i=1}^{M-1}{|a_i^j - a_j^0|} < \epsilon,
                        \sum_{i=M}^{\infty}{|a_i^j|} < 2\epsilon,
                        \sum_{i=M}^{M-1}{|a_i^0|} < 2\epsilon
                    \]
                    \[
                        \Rightarrow \sum_{1}^{\infty}{|a_i^j - a_i^0|} \le \sum_{i=1}^{M}{|a_i^j - a_i^0|} + \sum_{i=M}^{\infty}{a_i^0} < 5 \epsilon.
                    \]
                    So $x^j \rightarrow x^0$ in $(l^1, d)$.
                \end{itemize}        
            \end{proof}
        \end{enumerate}
    \end{example}
\end{definition}
\end{document}


