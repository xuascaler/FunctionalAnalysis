\documentclass{article}
\usepackage{amsthm}
\usepackage{amsfonts}

\title{Topological Vector Space}
\author{xuascaler}
\date{\today}

\newtheorem*{property}{Property}
\newtheorem*{definition}{Definition}
\newtheorem*{remark}{Remark}
\newtheorem*{example}{Example}
\newtheorem*{corollary}{Corollary}

\begin{document}
\maketitle

\section*{1. Local geometry of topological vector space}
Let $(X, \mathcal{F})$ be a topological space.
\begin{remark} \hfill
  \begin{itemize}
    \item A base for $\mathcal{F}$ is a subcollection 
    $\mathcal{F}' \subset \mathcal{F}$ s.t. 
    $\forall U \in \mathcal{F}, \exists V \in \mathcal{F}', V \subset U$.
    A base 
    $
      \mathcal{F}'$ determines $\mathcal{F} \Leftrightarrow
      \exists S, \forall U \subset \mathcal{F},
      U = \{\cup e_i \mid i \in S, e_i \in \mathcal{F}\}
    $.
    \item  A local base of $x$ is a subcollection 
    $\mathcal{F}_x' \subset \mathcal{F}_x
    \textnormal{ s.t }. \forall U \subset \mathcal{F}_x, \exists V \in \mathcal{F}_x' \textnormal{ s.t } V \subset U$.
    However, elements in $\mathcal{F}_x$ may be not union of elements in $\mathcal{F}_x'$.
  \end{itemize}
  \begin{example}
    $(X, d)$ is a metric space.
    \begin{itemize}
      \item $\mathcal{F}' = \{B(x, r) \mid x \in X, r > 0\}$ is a base.
      \item $\mathcal{F}' = \{B(x, r), \mid\}$ is a local at x.
      \item $\mathcal{F}' = \{B(x, \frac{1}{n}) \mid n \in \mathbb{N}\}$
      anohter local base at x, countable elements.
    \end{itemize}
  \end{example}
  Now let X be topological vector space. Last time we showed that 
  $\forall a \in X, \forall \alpha \neq 0$, the maps
  \begin{itemize}
    \item $T_a: X \rightarrow X, x \rightarrow x + a$
    \item $M_a: X \rightarrow X, x \rightarrow \alpha x$
  \end{itemize}
  are both homeomorphism. As a consequence, we see
  \begin{corollary}
    $
    \textnormal{A set }  A \subset X \textnormal{ is open} \Leftrightarrow
    a + A \textnormal{ is open}, \forall a \in X \Leftrightarrow
    \alpha A \textnormal{ is open}, \forall \alpha \neq 0.
    $ 
  \end{corollary}
\end{remark} 
  So the topological $\mathcal{F}$ is determined by any local base at $0$ 
  whose elements have special gemometric properties for topological vector space.
\begin{definition}
  $X$ is locally convex if there is a local base whose elements is convex.
\end{definition}
\begin{example}
  Normed Vector Space are locally convex since $\{B(0, r) \mid r > 0\}$ are convex.
  \begin{proof}
    $
    x, y \in B(0, r) \Leftrightarrow \|x\| < r, \|y\| < r \Leftrightarrow 
    \|\alpha x + (1 - \alpha)y\| \le \alpha\|x\| + (1 - \alpha)\|y\| \le r
    $
  \end{proof}
\end{example}
A set $E \subset X$ is absorbing if 
$
\forall x \in X, \exists \delta > 0 \textnormal{ s.t. } \delta x \in E 
\forall |\alpha| < \delta.
$(Obviously $0 = 0 * x \in E$)

\begin{property}
In a topological vector space, any neigborhood of $0$ is absorbing.
\begin{proof}
  Let $U$ be a neigborhood of 0. 
  $
  \forall x \in X, \textnormal{ the map } \mathbb{C} \rightarrow X: 
  \alpha \rightarrow \alpha x
  \textnormal{ is continuous. since it is the composition }
  \mathbb{R} \rightarrow \mathbb{R} \times X \rightarrow X: 
  \alpha \rightarrow (\alpha, x) \rightarrow \alpha x 
  \textnormal{ both of which are continuous. }
  $
  \begin{remark}
    $
    \textnormal{The function } F: X \rightarrow Y 
    \textnormal{ is continuous} \Leftrightarrow 
    \forall     
    $
    $
    Y' \subset Y, Y' \textnormal{ is open set }
    $
    the preimage of $Y'$ is open set. 
  \end{remark}
  So the pre-image of $U$ is an open set in $\mathbb{R}$, which obviously contains 0.
  So $\exists \delta$ s.t. $\forall |\alpha| < \delta, \alpha x \in U.$
\end{proof}
\end{property}
\begin{corollary}
  For any neigborhood of $0, X = \cup_{k=1}^{\infty}(kU)$
  \begin{proof}
    $
      \forall x \in X, \exists k, 
      \frac{1}{k} < \delta, \frac{1}{k} x \in U \Rightarrow x \in kU.
    $
  \end{proof}
\end{corollary}

\begin{definition}
  A set $E \subset X$ is symmetric if $E = -E$.
\end{definition}
\begin{property}
  $\forall U, 0 \in U$, one can find a symmetric neigborhood $V$ of 0 s.t.
  $V + V \subset U$.
  \begin{proof}
    Since $0 + 0 = 0$, and addition is continuous, 
    for the neigborhood $U$ of $0$, one can find neigborhoods $U_1, U_2$
    of $0$ s.t. $U_1 + U_2 \subset U$. 
    Take $V = U_1 \cap U_2 \cap -U_1 \cap -U_2$. 
    $V$ is symmetric and $0 \in V$. 
    $V \subset U_1, V \subset U_2, V + V \subset U$.
  \end{proof}
  \begin{remark}
    By iteration, one can find $V$ s.t. 
    $V + V + V + V \subset U$
  \end{remark}
\end{property}

\end{document}


