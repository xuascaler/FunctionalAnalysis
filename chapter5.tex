\documentclass{article}
\usepackage{amsthm}
\usepackage{amsfonts}

\title{Topological Vector Space}
\author{xuascaler}
\date{\today}

\newtheorem*{property}{Property}
\newtheorem*{definition}{Definition}
\newtheorem*{remark}{Remark}
\newtheorem*{example}{Example}
\newtheorem*{corollary}{Corollary}

\begin{document}
\maketitle

\section*{1. Local geometry of topological vector space}
Let $(X, \mathcal{F})$ be a topological space.
\begin{remark} \hfill
  \begin{itemize}
    \item A base for $\mathcal{F}$ is a subcollection 
    $\mathcal{F}' \subset \mathcal{F}$ s.t. 
    $\forall U \in \mathcal{F}, \exists V \in \mathcal{F}', V \subset U$.
    A base 
    $
      \mathcal{F}'$ determines $\mathcal{F} \Leftrightarrow
      \exists S, \forall U \subset \mathcal{F},
      U = \{\cup e_i \mid i \in S, e_i \in \mathcal{F}\}
    $.
    \item  A local base of $x$ is a subcollection 
    $\mathcal{F}_x' \subset \mathcal{F}_x
    \textnormal{ s.t }. \forall U \subset \mathcal{F}_x, \exists V \in \mathcal{F}_x' \textnormal{ s.t } V \subset U$.
    However, elements in $\mathcal{F}_x$ may be not union of elements in $\mathcal{F}_x'$.
  \end{itemize}
  \begin{example}
    $(X, d)$ is a metric space.
    \begin{itemize}
      \item $\mathcal{F}' = \{B(x, r) \mid x \in X, r > 0\}$ is a base.
      \item $\mathcal{F}' = \{B(x, r), \mid\}$ is a local at x.
      \item $\mathcal{F}' = \{B(x, \frac{1}{n}) \mid n \in \mathbb{N}\}$
      anohter local base at x, countable elements.
    \end{itemize}
  \end{example}
  Now let X be topological vector space. Last time we showed that 
  $\forall a \in X, \forall \alpha \neq 0$, the maps
  \begin{itemize}
    \item $T_a: X \rightarrow X, x \rightarrow x + a$
    \item $M_a: X \rightarrow X, x \rightarrow \alpha x$
  \end{itemize}
  are both homeomorphism. As a consequence, we see
  \begin{corollary}
    $
    \textnormal{A set }  A \subset X \textnormal{ is open} \Leftrightarrow
    a + A \textnormal{ is open}, \forall a \in X \Leftrightarrow
    \alpha A \textnormal{ is open}, \forall \alpha \neq 0.
    $ 
  \end{corollary}
\end{remark} 
  So the topological $\mathcal{F}$ is determined by any local base at $0$ 
  whose elements have special gemometric properties for topological vector space.
\begin{definition}
  $X$ is locally convex if there is a local base whose elements is convex.
\end{definition}
\begin{example}
  Normed Vector Space are locally convex since $\{B(0, r) \mid r > 0\}$ are convex.
  \begin{proof}
    $
    x, y \in B(0, r) \Leftrightarrow \|x\| < r, \|y\| < r \Leftrightarrow 
    \|\alpha x + (1 - \alpha)y\| \le \alpha\|x\| + (1 - \alpha)\|y\| \le r
    $
  \end{proof}
\end{example}
\begin{definition}
  A set $E \subset X$ is absorbing if 
  $
  \forall x \in X, \exists \delta > 0 \textnormal{ s.t. } \delta x \in E, 
  \forall |\alpha| < \delta.
$(Obviously $0 = 0 * x \in E$)
  
\end{definition}

\begin{property}
In a topological vector space, any neigborhood of $0$ is absorbing.
\begin{proof}
  Let $U$ be a neigborhood of 0. 
  $
  \forall x \in X, \textnormal{ the map } \mathbb{C} \rightarrow X: 
  \alpha \rightarrow \alpha x
  \textnormal{ is continuous. since it is the composition }
  \mathbb{R} \rightarrow \mathbb{R} \times X \rightarrow X: 
  \alpha \rightarrow (\alpha, x) \rightarrow \alpha x 
  \textnormal{ both of which are continuous. }
  $
  \begin{remark}
    $
    \textnormal{The function } F: X \rightarrow Y 
    \textnormal{ is continuous} \Leftrightarrow 
    \forall     
    $
    $
    Y' \subset Y, Y' \textnormal{ is open set,}
    $
    the preimage of $Y'$ is open set. 
  \end{remark}
  So the pre-image of $U$ is an open set in $\mathbb{R}$, which obviously contains 0.
  So $\exists \delta$ s.t. $\forall |\alpha| < \delta, \alpha x \in U.$
\end{proof}
\end{property}
\begin{corollary}
  For any neigborhood $U$ of $0, X = \cup_{k=1}^{\infty}(kU)$
  \begin{proof}
    $
      \forall x \in X, \exists k, 
      \frac{1}{k} < \delta, \frac{1}{k} x \in U \Rightarrow x \in kU.
    $
  \end{proof}
\end{corollary}

\begin{definition}
  A set $E \subset X$ is symmetric if $E = -E$.
\end{definition}
\begin{property}
  $\forall U, 0 \in U$, one can find a symmetric neigborhood $V$ of 0 s.t.
  $V + V \subset U$.
  \begin{proof}
    Since $0 + 0 = 0$, and addition is continuous, 
    for the neigborhood $U$ of $0$, one can find neigborhoods $U_1, U_2$
    of $0$ s.t. $U_1 + U_2 \subset U$. 
    Take $V = U_1 \cap U_2 \cap (-U_1) \cap (-U_2)$. 
    $V$ is symmetric and $0 \in V$. 
    $V \subset U_1, V \subset U_2, V + V \subset U$.
  \end{proof}
  \begin{remark}
    By iteration, one can find $V$ s.t. 
    $V + V + V + V \subset U$
  \end{remark}
\end{property}
\begin{definition}
  A neigborhood of $0$ in $X$ in balanced if $\alpha E \subset E$ for all $\alpha$ with $|\alpha| \le 1$.
  \begin{remark} \hfill
    \begin{enumerate}
      \item If $E$ is balanced, then $E$ is symmetric. since 
      \[-E \subset E, -(-E) \subset -E\].
      \item If $A, B$ are balanced, so is $A + B$.
    \end{enumerate}
  \end{remark}
  \begin{property}
    In a Topological vector space, any neigborhood of $0$ contains a balanced neigborhood of $0$.
    \begin{proof}
      Let $U$ be a neigborhood of $0$. By continuity and $0 \cdot 0 = 0$, 
      one can find $\delta > 0$ and neigborhood $V_1$ of $0$ 
      s.t. $\beta V_1 \subset U$ for any $\beta$ with $|\beta| < \delta$.
      Let $V = \cup_{0 < |\beta| < \delta}\beta V_1$, then 
      \begin{itemize}
        \item $V$ is open as union of open sets
        \item $V \subset U$ since each $\beta V_1 \subset U$.
        \item $V$ is balanced since $|\alpha| \le 1$, 
              $|\beta| < \delta \Rightarrow |\alpha \beta| < \delta$.
      \end{itemize}
    \end{proof}
  \end{property}
  \begin{corollary}
    Every topological vector space has a balanced local base.
    \begin{remark}
      Simlarly one can prove: any convex neigborhood of $0$ contains a balanced convex neigborhood of $0$.
      So any locally convex topological vector space has a balanced convex local base.
    \end{remark}
  \end{corollary}
  A subset $E \subset X$ is bounded if for any neigborhood $U$ of $0$ in $X$, $\exists s > 0$ s.t. 
  $\forall t > s$, we have $E \subset tU$.
  \begin{property}
    $E$ is balanced $\Leftrightarrow$ For any sequence $\{x_n\} \subset E$ and any scalar sequence $\alpha_n \rightarrow 0$,
    one has $\alpha_n x_n \rightarrow 0$
    \begin{proof} \hfill
      \begin{itemize}
        \item $\Rightarrow$, For any neigborhood $U$ of $0$,  
        we have $\exists s > 0, \forall t > s, E \subset tU$,
        Since $\alpha_n \rightarrow 0$, $\exists N, \forall n > N, |\alpha_n| < \frac{1}{t}$,
        we have $|\alpha_n|E \subset U \Rightarrow |\alpha_n|x_n \in U$. we have $|\alpha_n|x_n \rightarrow 0$.
        $\forall \delta > 0, \exists N, \forall n > N, ||\alpha_n|x_n - 0| = |\alpha_n x_n| = |\alpha_n x_n - 0| < \delta$.
        So $\alpha_n x_n \rightarrow 0$.
        \item $\Leftarrow$,
        If $E$ is not bounded, then for any neigborhood $U$ of 0,
        $\forall s > 0, \exists t > s, E \not \subset tU$.
        We fix $U$,
        now we construct $\{\alpha_n\}, \{x_n\}$.
        We choose $n \in \mathbb{N}, s = n, \exists t_n > n, \frac{E}{t_n} \not \subset U$,
        so $\alpha_n = \frac{1}{t_n}, x_n \in E, \frac{x_n}{t_n} \not \in U$.
        Now we have $\{\alpha_n \rightarrow 0, \alpha_n x_n \not \in U \Rightarrow \alpha_n x_n \not \rightarrow 0\}$.
        Contradiction!
      \end{itemize}
    \end{proof}
  \end{property}
  So in particular, if $X$ is a metric space, then $E$ is bounded iff $\exists C > 0, E \subset B(0, C)$.
  Not every topological vector space admit a bounded open set. In fact,
  \begin{property}
    If $V$ is bounded neigborhood of $0$, then for any sequence $\alpha_k \rightarrow 0$, the collection
    $\{\alpha_k V \mid k = 1, 2, 3...\}$ is a local base of $X$.
    \begin{proof}
      We construct a set $W = V \cup -V$, $V \subset W$ and $W$ is bounded and symmetric.
      For any neigborhood $U$ of 0, $\exists s > 0, \forall t > s, W \subset tU$.
      Since $\alpha_n \rightarrow 0 \Rightarrow \exists \alpha_n, \frac{1}{|\alpha_n|} > t$,
      we have $\exists \alpha_n, |\alpha_n| W \subset U$. 
      $W$ is symmetric, $\alpha_n W \subset U$ $V \subset W \Rightarrow \alpha_n V \subset U$.
      So ${\alpha_n V}$ is a local base.
    \end{proof}
  \end{property}
\end{definition}
\begin{definition}
  $X$ is locally bounded if $0$ has a bounded neigborhood.
\end{definition}
So any locally bounded TVS has a countable local base. According the next theorem, it must be metrizable.

\end{document}


