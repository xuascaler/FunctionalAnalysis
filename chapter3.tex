\documentclass{article}
\usepackage{amsthm}
\usepackage{amsfonts}
\usepackage{enumitem}

\title{The Baire category theorem}
\author{xuascaler}
\date{\today}

\newtheorem*{property}{Property}
\newtheorem*{definition}{Definition}
\newtheorem*{remark}{Remark}
\newtheorem*{example}{Example}
\newtheorem*{corollary}{Corollary}
\newtheorem*{theorem}{Theorem}

\begin{document}
\maketitle

\section*{Metric Topology}
Let $(X, d)$ be a metric space, and $A \subset X$ a subset.
\begin{definition} \hfill
    \begin{enumerate}
        \item A point $x \in A$ is called an interior point of $A$ if $\exists \epsilon > 0$ s.t.
        $B(x, \epsilon) = \{y \in X \mid d(y, x) < \epsilon\}$ lies in $A$.($\epsilon \textnormal{-neigborhood of x}$).
        \item $A$ is open if any point $x \in A$ is an interior point of $A$.
    \end{enumerate}
    \begin{example}
        For any $x \in X$ and any $r > 0$, $B(x, r)$ is open.
        \begin{proof}
            For any $y \in B(x, r)$, we have $d(x, y) < r$. Take any $0 < \epsilon < r - d(x, y)$.
            Then for any $z \in B(y, \epsilon)$,
            $d(z, x) \le d(z, y) + d(y, x) < \epsilon + d(x, y) < r$.
        \end{proof}
    \end{example}
    \begin{property} \hfill
        \begin{enumerate} 
            \item $\emptyset, X$ are open sets.
            \item If $\{A_\alpha\}$ are a collection (could be infinite, or even incountable) of open sets in $X$,
            so is $\cup_\alpha A_\alpha$
            \item If $A, B$ are open sets in $X$, so is $A \cap B$.(If $A_1, A_2, ..., A_n$ are open, so is $\cap_{i=1}{n}A_i$)
        \end{enumerate}
        \begin{proof} \hfill
            \begin{enumerate}
                \item Obvious.
                \item Suppose $x \in \cup_\alpha A_\alpha$, then $\exists \alpha, x \in A_\alpha$. Since $A_\alpha$ is open,
                $\exists \epsilon > 0, B(x, \epsilon) \subset A_\alpha$. It follows $B(x, \epsilon) \subset \cup_\alpha A_\alpha$.
                So $\cup_\alpha A_\alpha$ is open.
                \item Suppose $x \in A \cap B$, 
                then $\exists \epsilon1, \epsilon2$ s.t. $B(x, \epsilon1) \subset A, B(x, \epsilon2) \subset B$.
                Take $\epsilon = \min{\epsilon1, \epsilon2}$, Then $B(x, \epsilon) \subset A \cap B$.
                So $A \cap B$ is open.
                \begin{remark}
                    It could happen that the intersection of countable open sets is no longer open.
                    The simplest example is $A_n = (-\frac{1}{n}, \frac{1}{n} + 1)$, $\cap_{n=1}^{\infty}{A_N}$ is not open.
                \end{remark}
            \end{enumerate}
        \end{proof}
    \end{property}
\end{definition}
One can characterize convergence using open sets.
\begin{property}
    $\lim\limits_{i \rightarrow \infty}{x_i}=x_0$ if and only if for any open set $A$ containing $x_0$, 
    there exists $N$ s.t. $\forall i > N, x_i \in A$
    \begin{proof} \hfill
        \begin{itemize}
            \item Suppose $x_i \rightarrow x_0$ and $A$ is an open set containing $x_0$.
            Then $\exists \epsilon$ s.t. $\forall i > N, B(x_i, \epsilon) \subset A  \Rightarrow x_i \in B(x_0, \epsilon) \forall i > N$.
            So $\forall i > N, x_i \in A$.
            \item Suppose for any open set $A$ containing $x_0$, we can find $N$ s.t. $x_i \in A \forall i > N$,
            Then in particular $\forall \epsilon > 0, A = B(x_0, \epsilon) \forall i > N$,
            In other words, $\forall i > N, d(x_i, x_0) < \epsilon.$.
            So $x_i \rightarrow x_0$.
        \end{itemize}
    \end{proof}
\end{property}
\begin{definition}
    A subset $A \subset X$ is closed if $X \backslash A$ is open.
\end{definition}
One can easily convert properties of open sets to properties of closed sets.
\begin{example}
    For any $x \in A$ and any $r > 0$, $\overline{B}(x, r) = \{y \in X \mid d(y, x) \le r\}$ is closed.
    \begin{proof}
        % By the triangle inequalities, one can show as before that
        % $X \backslash \overline{B}(x, r) = \{y \in X \mid d(y, x) > r\}$ is open.
        we prove the set $X \backslash \overline{B}(x, r)$ is open.
        \[
            X \backslash \overline{B}(x, r) = \{y \in X \mid d(y, x) > r\}
        \]
        We claim that $\forall y \in X \backslash \overline{B}, \exists \epsilon > 0, B(y, \epsilon) \subset X \backslash \overline{B}$.
        We set $\epsilon = d(y, x) - r$.
        $\forall z \in B(y, \epsilon), d(x, z) + d(z, y) \ge d(x, y) \Leftrightarrow d(x, z) > d(x, y) - d(z, y) > d(x, y) - \epsilon = r$.
        So $d(x, z) > r, B(y, \epsilon) \subset X \backslash \overleftarrow{B}$,
        $X \backslash \overline{B}$ is open set.
    \end{proof}
\end{example}
A characterization of closed sets.
\begin{property}
    $A$ is closed iff for any sequence $x_n \in A, x_n \rightarrow x \in X, x \in A$.
    \begin{proof} \hfill
        \begin{itemize}
            \item Suppose $A$ is closed. $x_n \in A, x_n \rightarrow x \in X$. We want to show $x \in A$. 
            By contradiction: If $x \in X \backslash A$, one can find $\epsilon > 0, B(x, \epsilon) \subset X \backslash A$.
            i.e. $B(x, \epsilon) \cap A = \emptyset$.
            But $x_n \rightarrow x \Rightarrow \exists N, \forall n > N, x_n \in B(x, \epsilon)$.
            So $\forall n > N, x_n \notin A$. Contradiction!
            \item Suppose for any sequence $x_n \in A, x_n \rightarrow x \in X, x \in A$.
            We want to show $A$ is closed $\Leftrightarrow$ we want to show $X \backslash A$ is open.
            $\Leftrightarrow$ we want to show $\forall y \in X \backslash A$, 
            $\exists \epsilon > 0, B(y, \epsilon) \subset X \backslash A \Leftrightarrow B(y, \epsilon) \cap A = \emptyset$.
            Again by contradiction, suppose $\forall \epsilon > 0, B(y, \epsilon) \cap A \ne \emptyset$.
            Then choose $x_n \in B(y, \frac{1}{n}) \cap A$. Then $x_n \in A$ and $x_n \rightarrow y$. 
            so $y \in A$, contradicts with the fact $y \in X \backslash A$.
        \end{itemize}
    \end{proof}
\end{property}
\begin{definition}
    For any $A \subset X$, we define its closure to be the set 
    $\overline{A} = \{x \in X \mid \exists x_n \in A, x_n \rightarrow x\}$.
    \begin{example} \hfill
        \begin{itemize}
            \item $\mathbb{Q} \subset \mathbb{R}, \overline{\mathbb{Q}} = \mathbb{R}$.
            Since any real number is the limit of a sequence of rationals.
            \item $P([0, 1]) = \textnormal{ polynomials for } x \in [0, 1]$.
            The $P([0, 1]) \subset C([0, 1])$.
            In mathmatical analysis we learned that any continuous function is approximated uniformly by polynomials(e.g. Bernstain Polynomials),
            So if we use $d_0$ metric, the $\overline{P[0, 1]}=C([0,1])$.
        \end{itemize}
    \end{example}
    \begin{property}
        $\overline{A}$ is closed.
        \begin{proof}
            Suppose $x_n \in \overline{A}, x_n \rightarrow x \in X$. We want to show $x \in \overline{A}$.
            For any $n$, we choose $x_n \in \overline{A}$ s.t. $d(x_N, x) < \frac{1}{n}$.
            Since $x_N \in \overline{A}$, we can find an element in $A$,
            which we denoted by $y_n$, s.t. $d(y_n, x_N) < \frac{1}{n}$.
            Then $y_n \in A$ and $d(y_n, x) < d(y_n, x_N) + d(x_N, x) < \frac{2}{n}$.
            So $y_n \rightarrow x$, i.e. $x \in \overline{A}$.
            \begin{remark}
                If $B$ is closed, $A \subset B$, then $\overline{A} \subset B$.
                As a consequence, $\overline{A}$ is the smallest closed subset of $X$ which contains $A$.
            \end{remark}
        \end{proof}
    \end{property}
\end{definition}
\section*{2. The Baire category theorem}
\begin{definition}
    A subset $A \subset (X, d)$ is dense if $\overline{A} = X$.
    Equivalently, $\forall x \in X, \exists x_n \in A$ s.t. $x_n \rightarrow x$.
    \begin{example} \hfill
        \begin{itemize}
            \item $\mathbb{Q}$ is dense in $\mathbb{R}$, $\mathbb{R} \backslash \mathbb{Q}$ is also dense in $\mathbb{R}$.
            \item $P([0, 1])$ is also dense in $C([0, 1])$.
        \end{itemize}
    \end{example}
    \begin{property}
        $A \subset X$ is dense iff for any nonempty open subset $B \subset X$, $A \cap B \ne \emptyset$.
        \begin{proof} \hfill
            \begin{itemize}
                \item Suppose $A$ is dense, and $B \ne \emptyset$ is open.
                We choose $x \in B, \exists \epsilon > 0, B(x, \epsilon) \subset B$.
                Since $A$ is dense, one can find $\{y_n\}$ s.t. $y_n \rightarrow x$.
                So $\exists y \in A, d(y, x) < \epsilon \Leftrightarrow y \in A \cap B(x, \epsilon) \Leftrightarrow y \in A \cap (B, \epsilon)$.
                So $y \in A \cap B \Rightarrow A \cap B \ne \emptyset$.
                \item Suppose for any $B \ne \emptyset$ open, we have $A \cap B \ne \emptyset$.
                Then $\forall x \in X$, we can find $x_n \in B(x, \frac{1}{n})\cap A$.
                We get a sequence $\{x_n\} \in A, x_n \rightarrow x$. So $X = \overline{A}$, $A$ is dense. 
            \end{itemize}
        \end{proof}
    \end{property}
\end{definition}
\begin{definition}
    A subset $A \subset (X, d)$ is nowhere dense $\overline{A}$ contains no interior point.
    \begin{example} \hfill
        \begin{itemize}
            \item $\mathbb{Z}$ is no where dense in $\mathbb{R}$.
            \item The Cantor set is nowhere dense in $\mathbb{R}$
            \item $A=\{f \in C([0, 1]) \mid f(0) = 0\}$ is nowhere dense in $C([0, 1])$.
        \end{itemize}
    \end{example}
\end{definition}
\begin{definition}
    A subset $A \subset X$ is of first category if it is the union of countably many nowhere dense subsets.
    A subset $A \subset X$ is of second category if it is not of first cagtegory.
    \begin{example} \hfill
        \begin{itemize}
            \item $\mathbb{Q}$ is of first category.
            \item By the next theorem and it's corollary, $\mathbb{R} \backslash \mathbb{Q}$ is $2^{nd}$ category.
        \end{itemize}
    \end{example}
\end{definition}
\begin{theorem}
    Let $(X, d)$ be a complete metric space. Then the intersection of any countable collection of dense open subsets of $X$
    is still dense in $X$, but not necessary open.
    \begin{proof}
        Let $A_1, A_2, ..., A_n$ be a sequence of dense open subsets of $X$.
        Take any nonempty open set  $B \subset X$, we want to show $(\cap_{i=1}^\infty A_i) \cap B \ne \emptyset$.
        \begin{itemize}
            \item $A_1 \subset X$ is dense open, we see $A_1 \cap B \ne \emptyset$ and $A_1 \cap B$ is open.
            So in particular, we can find $x_1 \in X, r_1 > 0, \overline{B(x_1, y_1)} \subset A_1 \cap B$. 
            \item We continue by induction.
            Suppose we have chosen $x_{n-1}, r_{n-1}$ s.t.
            \[
                \overline{B(x_{n-1}, r_{n-1})} \subset A_{n-1} \cap B(x_{n-2}, r_{n-2})
            \]
            Since $A_n$ is dense open, $A_n \cap B(x_{n-1}, y_{n-1})$ is non-empty and open.
            So one can find $x_n \in X$, $r_n > 0$ s.t.
            \[
                \overline{B(x_n, r_n)} \subset A_n \cap B(x_{n-1}, r_{n-1})
            \]
            Note that we can always take $r_n < \frac{1}{n}$.
            \item We claim that $\{x_n\}$ is a Cauchy sequence. In fact, $\forall N$ and
            $n, m > N$, $B(x_n, y_n) \subset B(x_N, y_N)$ and $B(x_m, y_m) \subset B(x_N, y_N)$.
            In particular, $x_n, x_m \in B(x_N, y_N)$.
            So $d(x_n, x_m) \le d(x_n, x_N) + d(x_N, x_m) < 2r_N < \frac{2}{N}$,
            So $\{x_n\}$ is Cauchy.
            \item By completeness, we can find $x_0 \in X$ s.t. $x_n \rightarrow x_0$.
            Since $\overline{B(x_n, y_n)} \subset B(x_{n-1}, r_{n-1})$, we have 
            \[
                x_0 \in \overline{B(x_n, r_n)} \subset A_n \cap B(x_{n-1}, r_{n-1}) \subset ... \subset A_n \cap B.
            \]
            It follows that $x_0 \in \cap_{n=1}^{\infty}{(A_n \cap B)} = (\cap_{n=1}^{\infty}{A_n}) \cap B$.
            So $(\cap_{n=1}^{\infty}A_n) \cap B \ne \emptyset$.
        \end{itemize}
    \end{proof}
    \begin{corollary}
        Any complete metric space is of $2^nd$ category.
        \begin{proof}
            Assume $X$ is of $1^nd$ category which means $X=\cap_{i=1}^{\infty}$ where $A_i$ is nowhere dense.
            Then $X=\cap_{i=1}^{\infty}\overline{A_i}$, and thus 
            \[
                \cap_{i=1}^{\infty}{X \backslash \overline{A_i}} = X \backslash \overline{A_0} 
                                                                     \backslash \overline{A_1} 
                                                                     \backslash ...
                                                                     \backslash \overline{A_n} = \emptyset
            \]
            Each $X \backslash \overline{A_i}$ is open because $\overline{A_i}$ is closed.
            Each $X \backslash \overline{A_i}$ is dense because for all open set $B$, 
            if $B \cap (X \backslash \overline{A_i}) = \emptyset$, then $B \subset \overline{A_i}$,
            $\overline{A_i}$ has no interior point, $B \subset \overline{A_i}$ can never happen.
            So $B \cap (X \backslash \overline{A_i}) \ne \emptyset$, $X \backslash \overline{A_i}$ is dense.
            By the previous theorem, $\cap_{i=1}^{\infty}{(X \backslash \overline{A_i})}$ is dense, and thus not $\emptyset$.
            Contradiction!
        \end{proof}
    \end{corollary}
\end{theorem}
\end{document}
