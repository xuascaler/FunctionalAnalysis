
\documentclass{article}
\usepackage{amsthm}
\usepackage{amsfonts}

\title{Fr\'echet Spaces}
\author{xuascaler}
\date{\today}

\newtheorem*{property}{Property}
\newtheorem*{definition}{Definition}
\newtheorem*{remark}{Remark}
\newtheorem*{example}{Example}
\newtheorem*{theorem}{Theorem}
\newtheorem*{corollary}{Corollary}

\begin{document}
\maketitle

\section*{1 Topology defined by semi-norms}
Let's start with the example we mentioned at the begining of Lecture4.
\[
    X = C_0^0(\mathbb(R)) = \{ \textnormal{compactly supported continous functions on } \mathbb(R)\}
\]
Then as we have seen, $X = \cup_{n=1}^{\infty}X_n$, where
\[
    X_n = C_0^0([-n, n]) = \{ \textnormal{compactly supported continous functions on } (-n, n), f(n) = f(-n) = 0\}
\]
The topology of $X_n$, as a closed subspace of $C^0([-n, n])$, is determined by open sets
\[
    B(f, r) = \{ g \in C^0_0([-n, n]) \mid \sup_{x \in [-n, n]}|f(x) - g(x)| < r \}
\]
In particular, if we let $p_n(f) = sup_{x\in[-n, n]}{|f(x)|}$, then a local base for this topology is $\{f \mid p_n(f) < \frac{1}{k}\}$.
We would like to assign a topology to $X$, so that as a subspace of $X$, the space $X_n$ has the same topology mentioned above.
The easiest way is to let 
\[U_{n}^{k} = \{f \in C_0^0(\mathbb{R}) \mid p_n(f) = \sup_{x \in [-n, n]}{|f(x)| < \frac{1}{k}}\}\]
and take $\mathcal{F}$ be the topology generated by $U_{n,k}$.
\begin{remark} \hfill
    \begin{itemize}
        \item As a function on $C_0^0(\mathbb(R))$, $p_n$ is not a norm, although it is a norm on $C_0^0{[-n, n]}$.
        Since one can easily find a function $f \in C_0^0(\mathbb{R}), f \ne 0, p_n(f) = 0$.
        \item under this topology, $(X, \mathcal{F})$ is a locally convex TVS.
        \item There are many other spaces of this type.
    \end{itemize}
\end{remark}
\begin{definition}
    A semi-norm on a vector space $X$ is a function $p: X \rightarrow [0, + \infty)$ s.t.
    \begin{enumerate}
        \item $p(x + y) \le p(x) + p(y)$ (subadditivity)
        \item $p(\alpha x) = |\alpha| p(x)$(positive homogenerity)
    \end{enumerate}
    \begin{definition}
        A family of semi-norms, $\{p_{\lambda}\}$ on $X$ is called seperating if $ \forall x \ne 0, \exists \lambda, p_{\lambda}(x) \ne 0$.
    \end{definition}
    \begin{example}
        $p_n(f) = \sup_{x \in [-n, n]}{|f(x)|}$ defines a countable family of seperating semi-norms on $C_0^0(\mathbb{R})$.
    \end{example}
\end{definition}
\begin{theorem}
    Let $\mathcal{P} = \{p_{\lambda}\}$ be a seperating family of semi-norms on a vector space $X$. 
    For each $p_{\lambda} \in \mathcal{P}$ and each $k \in \mathbb{N}$ we let $U_{\lambda, k} = \{x \in X \mid p_{\lambda}{x} < \frac{1}{k}\}$ \\
    Let $\mathcal{B} = \{ \textnormal{the collection of all finite intersection of sets of the form } U_{\lambda, k}\}$. \\
    Let $\mathcal{F} = \{\textnormal{the translation invariant topology on $X$ that has $\mathcal{B}$ as a local base at $0$}\}$.
    Then
    \begin{enumerate}
        \item $(X, \mathcal{F})$ is a topological vector space.
        \item $\mathcal{B}$ is a convex balanced local base.
        \item Each $p_{\lambda} \in \mathcal{P}$ is continuous.
        \item A set $E \subset X$ is bounded iff each $p_{\lambda} \in \mathcal{P}$ is bounded on $E$.
    \end{enumerate}
    \begin{proof} \hfill
        \begin{enumerate}
            \item 
            \begin{itemize}
                \item $\mathcal{F}$ is Hausedoff. It is enough to seperate $0$ and $x \ne 0$.
                One just choose $p_{\lambda} \in \mathcal{P}$ s.t. $p_{\lambda}(x) \ne 0$.
                We denote $p_{\lambda}(x) = \epsilon$. Take $k$ large s.t. $\frac{1}{k} < \frac{\epsilon}{2}$.
                Then $U_{\lambda, k}$ is a neigborhood of $0$.
                $x + U_{\lambda, k}$ is a neigborhood of $x$, and $U_{\lambda, k} \cap (x + U_{\lambda, k}) = \emptyset$.
                So $\mathcal{F}$ is Hausedoff.
                If $y \in U_{\lambda, k} \cap (x + U_{\lambda, k})$, then $\exists z \in U_{\lambda, k}$ s.t. $y = x + z \Rightarrow x = y - z$.
                So $p_{\lambda}(x) = p_{\lambda}(y - z) \le p_{\lambda}(y) + p_{\lambda}(z) < \frac{1}{k} + \frac{1}{k} = \frac{2}{k} < \epsilon$
                Contrdiction! we know $p_{\lambda}(x) = \epsilon$!
                \item Vector addition is continuous. Let $U$ be any open neigborhood of $x+y$. By definition of $\mathcal{F}$, 
                one can choose $p_{\lambda_1}, ..., p_{\lambda_l}$ and $k_1, ..., k_l$ s.t.
                $x + y + (U_{\lambda_1, k_1} \cap ... \cap U_{\lambda_l, k_l}) \subset U$.
                Now let 
                \[U_1 = x + U_{\lambda_1, 2k_1} \cap ... \cap U_{\lambda_l, 2k_l}\textnormal{(open neigborhood of x)}\]
                \[U_2 = y + U_{\lambda_1, 2k_1} \cap ... \cap U_{\lambda_l, 2k_l}\textnormal{(open neigborhood of y)}\]
                Then $U_1 + U_2 \subset U$.
                If $z_1, z_2 \in U_{\lambda_1, 2k_1} \cap ... \cap U_{\lambda_l, 2k_l}$, i.e. 
                $\forall 1 \le i \le l, p_{\lambda_i}(z_1) < \frac{1}{2k_i}, p_{\lambda_i}(z_2) < \frac{1}{2k_i}$, 
                then $\forall 1 \le i \le l, p_{\lambda_i}(z_1 + z_2) < \frac{1}{k_i}$.
                \item Scalar multiplication is continuous. Let $U$ be a neigborhood of $\alpha x$, so as above,
                \[ \alpha x + U_{\lambda_1, k_1} \cap ... \cap U_{\lambda_l, k_l} \subset U \]
                for some $p_{\lambda_1}, ..., p_{\lambda_l} \in \mathcal{P}$ and $k_1, ..., k_l \in \mathbb{N}$.
                \begin{itemize}
                    \item case 1. $\alpha = 0$. We choose $A > max(p_{\lambda_1}(x), ..., p_{\lambda_l}(x))$,
                    then for $\delta < \min{\{\frac{1}{2Ak_1}, ..., \frac{1}{2Ak_l}, 1\}}$, $(-\delta, \delta)\cdot(x + U_{\lambda_1, 2k_1} \cap ...\cap U_{\lambda_l, 2k_l}) \subset U$
                    \[\forall e \in (-\delta, \delta) \cdot x, \forall 1 \le i \le l, p_{\lambda_i}(e) < \frac{1}{2k_i}\]
                    So \[(-\delta, \delta) \cdot x \subset U_{\lambda_1, 2k_1} \cap ...\cap U_{\lambda_l, 2k_l}\]
                    $|\delta| \le 1 \Rightarrow (-\delta, \delta) \cdot U_{\lambda_1, 2k_1} \cap ...\cap U_{\lambda_l, 2k_l} \subset U_{\lambda_1, 2k_1} \cap ...\cap U_{\lambda_l, 2k_l}$
                    \[(-\delta, \delta)\cdot(x + U_{\lambda_1, 2k_1} \cap ...\cap U_{\lambda_l, 2k_l}) \subset U_{\lambda_1, k_1} \cap ...\cap U_{\lambda_l, k_l} \subset U\]
                    \item case 2. $\alpha \ne 0$. We choose 
                    \[A > \max{\{p_{\lambda_1}(x), ..., p_{\lambda_l}(x),\frac{1}{3k_1|\alpha|}, ..., \frac{1}{3k_l|\alpha|}\}}\]
                    \[\delta < \min{\{\frac{1}{3k_1A}, ..., \frac{1}{3k_nA}, 1}\}\]
                    Then
                    \[ (\alpha - \delta, \alpha + \delta) \cdot (x + U_{\lambda_1, 3k_1|\alpha|} \cap ... \cap U_{\lambda_l, 2k_l|\alpha|}) \subset U\]
                \end{itemize}
            \end{itemize}
            \item By definition $\mathcal{B}$ is a local base for $\mathcal{F}$.
            To prove $U_{\lambda_1, k_1} \cap ... \cap U_{\lambda_l, k_l}$ is convex and balanced,
            it is enough to prove $U_{\lambda, k}$ is convex and balanced.
            It is balanced by positive homogenerity. It is convex since $\forall x, y \in U_{\lambda, k}$.
            $\forall 0 \le \alpha \le 1$,
            \[
                p_{\lambda}(\alpha x + (1 - \alpha)y) \le \alpha p_{\lambda}(x) + (1-\alpha)p_{\lambda}(y) < \alpha \frac{1}{k} + (1 - \alpha) \frac{1}{k} = \frac{1}{k}
            \]
            \item By definition each $p_{\lambda}$ is continuous at $0$.
            The continuity of $p_{\lambda}$ at $x$ follows from $p_{\lambda}(x + U_{\lambda, k}) \subset (p_{\lambda}(x) - \frac{1}{k}, p_{\lambda}(x) + \frac{1}{k})$.
            If $y \in U_{\lambda, k}$, then 
            \[p_{\lambda}(x + y) \le p_{\lambda}(x) + p_{\lambda}(y) < p_{\lambda}(x) + \frac{1}{k}\]
            \[p_{\lambda}(x + y) + p_{\lambda}(-y) \ge p_{\lambda}(x) \Rightarrow p_{\lambda}(x + y) \ge p_{\lambda}(x) - p_{\lambda}(-y) > p_{\lambda}(x) - \frac{1}{k}\]
            \item Suppose $E$ is bounded, and $p_{\lambda} \in \mathcal{P}$. Then $\exists t > 0$ s.t. $E \subset tU_{\lambda, 1}$ i.e. $\frac{1}{t}E \subset U_{\lambda, 1}$
            So $\forall x \in E, p_{\lambda}(x) < t$.
            Suppose each $p_{\lambda}$ is bounded on $E$. Then for any neigborhood $U_{\lambda_1, k_1} \cap ... \cap U_{\lambda_l, k_l} \subset U$ of $0$,
            one pick $t_1, ..., t_l$ s.t. $\forall x \in E, p_{\lambda_i}(x) < t_i$. 
            Then for $t > max(t_1 k_1, ..., t_l k_l)$, we have $E \subset t(U_{\lambda_1, k_1} \cap ... \cap U_{\lambda_l, k_l})$,
            since $p_{\lambda_i}(\frac{x}{t}) = \frac{1}{t} p_{\lambda_i}(x) < \frac{1}{t}t_i < \frac{1}{k_i}$.
        \end{enumerate}
    \end{proof}
\end{theorem}
\section*{2 Fre\'chet Spaces}
In many applications, as in the case of $C_0^0(\mathbb(R))$, the topology is defined by a countable sequence of semi-norms.
As a result, the local base $\mathcal{B}$ contains only countable elements, and thus $\mathcal{F}$ is metrizable. In fact,
we can explicitly write down a translation-invariant metric in this case.
\begin{property}
    If the topology on $X$ is defined by a seperating sequence $\{p_n\}$ of semi-norms,
    then for any sequence of positive members $\{c_i\}$ that tends to $0$,
    \[
        d(x, y) = \max_{i}\frac{c_i p_i (x-y)}{1 + p_i(x-y)}
    \]
    is a compatible translation-invariant metric on $X$.
\end{property}
\end{document}