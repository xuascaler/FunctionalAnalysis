\documentclass{article}
\usepackage{amsthm}
\usepackage{amsfonts}
\usepackage{enumitem}

\title{Topology vector spaces}
\author{xuascaler}
\date{\today}

\newtheorem*{property}{Property}
\newtheorem*{definition}{Definition}
\newtheorem*{remark}{Remark}
\newtheorem*{example}{Example}
\newtheorem*{corollary}{Corollary}
\newtheorem*{theorem}{Theorem}

\begin{document}
\maketitle

\section*{1. Topology spaces}
We have seen many important example of Banach spaces, 
or more generally examples of vector spaces with a metric structure.
However, there are also examples of important spaces 
whose natural structure does not follow from a complete metric.
\begin{example}
    $X = C_0^0(\mathbb{R}) = \{ \textnormal{compactly supported continuous function on } \mathbb{R}\}$
    If we let
    \[
        X_n = C_0^0(\bigl[-n, n \bigr]) = \{f \in C_0^0(\mathbb{R}): supp(f) \subset \bigl[ -n, n\bigr]\},
    \]
    \[
        supp(f)=\overline{\{x \mid f(x) \ne 0\}}
    \]
    \begin{itemize}
        \item $X = \cap_{n=1}^{\infty}{X_n}$
        \item $X_n \subset C^0(\bigl[-n, n\bigr])$ is closed.(Banach space)
        \item $X_n$ is nowhere dense in $C^0(-n, n)$(and in $C^0(\bigl[-m,m\bigr])$ for $m \ge n$).
    \end{itemize}
    Of course any reasonable structure in 
    $C_0^0(\mathbb{R})$ 
    should give the subsapce $C_0^0(\bigl[ -n, n \bigr])$ natural Banach space structure.\\
    As a consequence of the Baire category theorem, one cannot endow $C_0^0(\mathbb{R})$ with
    a complete metric whose induced topology is the natural one that we are interesting. \\
    So we need to study structures that are more general than the metric structure.
\end{example}
\begin{definition}
    A topological space is a set $X$ with a collection $\mathcal(F)$ of subsets of $X$,
    ($\mathcal{F}$ is called topology, and those element in $\mathcal{F}$ are called open sets),
    such that
    \begin{enumerate}
        \item $X, \emptyset \in \mathcal{F}$
        \item If $A_{\alpha} \in \mathcal{F}$, then $\cup_{\alpha}{A_{\alpha}} \in \mathcal{F}$
        \item If $A_1, A_2 \in \mathcal{F}$, then $A_1 \cap A_2 \in \mathcal{F}$
    \end{enumerate}
    \begin{example}
        \begin{itemize}
            \item $\mathcal{F} = \{X, \emptyset\}$ is called the weakest topology on $X$.
            \item $\mathcal{F} = \{ A | A \subset X \}$.
            \item The metric topology on $(X, d)$ is a topology.
        \end{itemize}
    \end{example}
\end{definition}
\begin{definition}
    A topological space is Housdorff if for any $x \ne y$, there exists neigbourhoods
    $U$ of $x$, $V$ of $y$ such that $U \cap V = \emptyset$.
    \begin{remark} \hfil
        \begin{enumerate}
            \item In a Housdorff space, the limit of a convergent sequence is unique.
            \begin{proof}
                Suppose $x_n \rightarrow x, x_n \rightarrow y, y \ne x$,
                take $U, V$ as above. then $\forall n > N, x_n \in U$,
                because $U \cap V = \emptyset \Leftrightarrow \forall n > N, x_n \not\in V$.
                $x_n \not\rightarrow y$.
            \end{proof}
            \item Any single point set $\{x\}$ is closed in a Housdorff space.
            \begin{proof}
                $\forall y \ne x$, we can find a neigbourhood $V_y$ of $y$ s.t. 
                $ x \not\in V_y$, so $X\backslash\{x\} = \cup_{y \ne x}{V_y}$ is open.
            \end{proof}
        \end{enumerate}
    \end{remark}
\end{definition}
Now let $(x, \mathcal{F})$ be a topological space.
\begin{definition} \hfil
    \begin{enumerate}
        \item A subcollection $\mathcal{F}' \subset \mathcal{F}$ is called a base for $\mathcal{F}$
        if any open set $U \in \mathcal{F}$ is the union of soem members in $\mathcal{F}'$.
        \item A subcollection $\mathcal{F_x}' \subset \mathcal{F_x}$ is called a base at $x$ 
        if every neigbourhood of $x$ contains an element of $\mathcal{F_x}'$.
        (But not necessary union of elements in $\mathcal{F}'$)
    \end{enumerate}
    \begin{example} \hfill
        \begin{itemize}
            \item $\mathcal{F}' = \{B(x, r) \mid x \in X, r > 0\}$ for a base for the metric topology on $(X, d)$.
            \item $\mathcal{F}_x' = \{B(x, r)| r > 0\}$ is a local base at $x$.
            \item $\mathcal{F}_x''= \{B(x, \frac{1}{n}) \mid n \in \mathbb(N)\}$ is a local base at $x$ 
            containing only countable many elements.
        \end{itemize}
    \end{example}
    \begin{remark} \hfil
        \begin{itemize}
            \item difference bases may generate the same topology.
            \item If $\mathcal{F'}$ is a base of $\mathcal{F}$, then $\mathcal{F}$ is the topology generated by $\mathcal{F}'$.
        \end{itemize}
    \end{remark}
\end{definition}
Now let $(X, \mathcal{F})$ and $(Y, \mathcal{G})$ be topology spaces. let
\[
    \mathcal{S} = {U \times V \mid U \in \mathcal{F}, V \in \mathcal{G}}
\]
Then $\mathcal{S}$ is collection of subsets in $X \times Y$.
\begin{definition}
    The topology generated by $\mathcal{S}$ is called the product topology on $X \times Y$.
    \begin{example}
        The usual topology on $\mathbb{R}^2$ is the product topology of the usual topology on $\mathbb{R}$,
        since any open subset in $\mathbb{R}^2$ is the union of "open rectangles".
    \end{example}
\end{definition}
Let $X, Y$ be topology spaces.
\begin{definition} \hfil
    \begin{enumerate}
        \item A map $f: X \rightarrow Y$ is called continuous at $x \in X$ if the inverse image of every
        open neigbourhood of $f(x)$ contains an open neigbourhood of $x$.
        \item $f$ is continuous on $X$ if it is continuous at every $x \in X$, in other words,
        $\forall V \in \mathcal{G}$, one has $f^{-1}(V) \in \mathcal{F}$.
    \end{enumerate}
    \begin{property}
        If $f: X \rightarrow Y$ is continuous at $x$, and $x_n \rightarrow x$, then $f(x_n) \rightarrow f(x)$.
        \begin{proof}
            For any neigbourhood $V$ of $f(x)$, the inverse image $f^{-1}(V)$ is a neigbourhood of $x$.
            So for any neigbourhood $V$, we can find $N$ s.t. $\forall n > N, x_n \in f^{-1}(V)$.
            $V$ can be any neigbourhood, so it can be any small.
            So $\forall n > N, f(x_n) \in V, f(x_n) \rightarrow f(x)$.
        \end{proof}
    \end{property}
\end{definition}
\begin{definition}
    A map $f: X \rightarrow Y$ is a homeomophism if it is continuous, invertable and the inverse is also continuous.
\end{definition}

\section*{Topological Vector Spaces}
Roughly speaking, a topological vector space is a vector space endowed with a topology so that the vector space operations
(vector addition, scalar multipliation) are compatable with the topological structure(i.e are continuous).
\begin{definition}
    Let $X$ be a vector space endowed with a Housdorff topology(some books do not require this) $\mathcal{F}$.
    It is said to be topological vector space if the mappings
    \[
        X \times X \rightarrow X, (x, y) \rightarrow x + y \\
    \]
    \[
        \mathbb{R}(\textnormal{or } \mathbb{C}) \times X \rightarrow X, (\alpha, x) \rightarrow \alpha x
    \]
    are continuous.(We use product topology on $X \times X, \mathbb{R} \times X$) \\
    By definition, the continuity of vector addition and scalar multipliation means
    \begin{itemize}
        \item $\forall x \in X, y \in X, \forall V \in \mathcal{F}_{x + y}, \exists U_x \in \mathcal{F}_x, \exists U_y \in \mathcal{F}_y$
        s.t. $U_x + U_y \subset V$.
        \item $\forall \alpha \in \mathbb{R}, \forall V \in \mathbb(F)_{\alpha x}, \exists \epsilon > 0, U_x \in \mathcal{F}_x$
        s.t. $(\alpha - \epsilon, \alpha + \epsilon) \cdot U_x \subset V$.
    \end{itemize}
    \begin{remark} \hfill
        \begin{itemize}
            \item For any $A, B \subset X$, we denote $A + B = \{ x + y \mid x \in A, y \in B\}$
            \item For any $I \subset \mathbb{R}, A \subset X$, we denote $I \cdot A = \{ \alpha x \mid \alpha \in I, x \in A\}$
        \end{itemize}
        \begin{example} \hfill
            \begin{itemize}
                \item example $+$
                \[ A = \{(x, 0) \mid -1 \le x \le 0\}, B = \{(1, y) \mid -1 \le y \le 1\} \]
                \[A + B = \{(x, y) \mid 0 \le x \le 2, -1 \le y \le 1\}\]
                \item example $\cdot$
                \[A = \{(1, 0), (2, 0)\} \]
                \[2A = \{(2, 0), (4, 0)\} \ne A + A = \{(2, 0), (3, 0), (4, 0)\}\]
            \end{itemize}
        \end{example}
        \begin{itemize}
            \item For any $a \in X$, one has a $translation operator$
            \[
                T_a: X \rightarrow X, x \rightarrow T_a(x) = a + x
            \]
            \item For any $0 \ne \alpha \in \mathbb{R}$, one has a multipliation operator.
            \[
                M_\alpha : X \rightarrow X, x \rightarrow M_\alpha(x) = \alpha x
            \]
        \end{itemize}
        \begin{property}
            For any $a \in X$ and any $0 \ne \alpha \in \mathbb{R}$, $T_a$ and $M_\alpha$ are homeomorphisms.
            \begin{proof}
                $T_a$ and $M_{\alpha}$ are both invertable, with inverse $T_{-a}$ and $M_{\frac{1}{\alpha}}$ respectively.
                Moreover, they are all continuous according to the continuity of vector addition and scalar multipliation.
            \end{proof}
        \end{property}
        \begin{corollary}
            A subsset $A$ is open if and only if $a + A$ is open. 
            So $\mathcal{F}$ is determined byany local base $\mathcal{F}_{0}'$ at $0$.
        \end{corollary}
    \end{remark}
\end{definition}
\end{document}